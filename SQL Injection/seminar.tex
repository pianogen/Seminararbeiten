\documentclass{article}
\usepackage[latin1]{inputenc}
\usepackage[ngerman]{babel}
\usepackage{graphicx}
\usepackage{enumitem}
\usepackage{multirow}
\usepackage{threeparttable}
\usepackage{lscape}
\usepackage{fancyhdr}
\usepackage[toc]{glossaries}
\usepackage{url}
\usepackage{pdfpages}
\makeglossaries
\pagestyle{fancy}
\fancyhead[LE,RO]{\slshape \rightmark}
\fancyhead[LO,RE]{\slshape \leftmark}
\fancyfoot[C]{\thepage}
\fancyfoot[L]{Gennaro Piano}
\fancyfoot[R]{\today}
\renewcommand{\footrulewidth}{0.4pt}

\begin{document}

\title{Seminar: Analyse und Angriffe auf Netzwerke  \\ SQL Injection}
\author{Gennaro Piano}
\date{\today}

\maketitle

\begin{figure}[htbp]
		\centering
	\includegraphics[width=6cm]{img/logo.jpg}
		\label{fig:logo}
\end{figure}

\newpage \thispagestyle{empty}
\tableofcontents
\newpage \thispagestyle{empty}
\listoffigures
\newpage \thispagestyle{empty}
\listoftables
\newpage
\section{Einleitung}
Dieses Seminar behandelt das Thema Analyse und Angriffe auf Netzwerk. Eine weitverbreitete Art um in einem Netzwerk schaden anzurichten, sind sogenannte SQL Injections. Dieses Projekt befasst sich mit SQL Injections. Dieses Kapitel erkl�rt den Begriff SQL Injection und befasst sich mit der genauen Aufgabenstellung des Projekts.
\subsection{Ausgangslage}
SQL-Injections sind eine h�ufige Angriffsart, weil sie einfach durchzuf�hren sind und sehr effizient sein k�nnen. SQL-Injections sind eingeschleuste SQL Abfragen, welche einem nicht autorisiertem Benutzer Zugriff zur Datenbank gew�hren. Somit kann ein nicht autorisierter Benutzer Daten aus der Datenbank lesen oder gar �ndern und l�schen. 
\subsection{Ziel der Arbeit}
Das Ziel dieser Arbeit ist es aufzuzeigen, was mit SQL Injections alles m�glich ist und wie man Anwendung gegen SQL Injections sch�tzen kann. Der Seminarbericht soll die Analyse dieser Schwachstelle beinhalten. Nebenbei soll eine Testumgebung entwickelt werden, welche aufzeigt wie Datenbank von Dritten manipuliert werden k�nnen und wie man sich vor solchen Angriffen sch�tzen kann.
\subsection{Aufgabenstellung}
Es werden zwei grafisch identische Websites erstellt, welche an die gleiche Datenbank angebunden sind. \newline \newline
Eine Website ist gegen SQL Injections gesch�tzt, die andere nicht. \newline \newline
Auf beiden Websites wird versucht anhand von SQL Injections Daten aus der Datenbank zu lesen und zu �ndern.
\subsection{Erwartete Resultate}
Als Resultat wird eine Dokumentation erwartet, welche die Resultate dokumentiert und wie man sich gegen SQL Injections sch�tzen kann.
\subsection{Sourcecode}
Der Sourcecode der Websites wird aus Platzgr�nden nicht in die Dokumentation eingef�gt, jedoch kann er direkt aus dem GitHub Repository heruntergeladen werden. Das Repository befindet sich unter \newline https://github.com/pianogen/Seminararbeit/SQLInjections.
\subsection{Bemerkung}
Der praktische Teil dieser Arbeit wird nur zur Analyse der SQL Statements verwendet und wird nie produktiv genutzt, somit wird das gesamte Testverfahren der Applikation nicht ber�cksichtigt.
\section{SQL Injections}
\subsection{Allgemein}
\subsection{Attacken}
\subsection{Schutz}
\end{document}