\documentclass{article}
\usepackage[latin1]{inputenc}
\usepackage[ngerman]{babel}
\usepackage{graphicx}
\usepackage{fancyhdr}
\pagestyle{fancy}
\fancyhead[LO]{Seminar: Netzwerksicherheit}
\fancyhead[RO]{ZHAW}
\renewcommand{\headrulewidth}{0.4pt}
\fancyfoot[C]{\thepage}
\fancyfoot[L]{Gennaro Piano}
\fancyfoot[R]{\today}
\renewcommand{\footrulewidth}{0.4pt}

\begin{document}
\section*{Network Access Protection}
\subsection*{Ausgangslage}
Das Unternehmen KMU IT Management AG sucht ein Produkt, welches infizierte Computer isoliert und somit keine Gefahr f�r andere Ger�te darstellt. Vor ein paar Jahren wurde das Produkt FreeNAC evaluiert, jedoch entsprach dies nicht den Vorstellungen des Unternehmens.
\subsection*{Ziel der Arbeit}
Dieses Projekt soll das Network Access Protection von Microsoft analysieren. Es soll die St�rken und Schw�chen aufzeigen. Als Resultat wird eine Empfehlung erwartet, ob Network Access Protection implementiert werden soll oder nicht.
\subsection*{Network Access Protection}
Network Access Protection ist ein Produkt von Microsoft. Es verhindert, dass Clients, die ungen�gend gesch�tzt sind, das Firmennetz besch�digen k�nnen.
Ungen�gend gesch�tzten Clients wird limitierter Zugriff ins Firmennetzwerk gew�hrt. Clients mit limitiertem Zugriff k�nnen nur Server kontaktieren, die sie aktualisieren und m�gliche Sicherheitsl�cken schliessen k�nnen. Falls dieser Vorgang erfolgreich abl�uft erh�lt der Client dadurch vollen Zugriff.
Der limitierte Zugriff kann durch f�nf verschiedene Methoden implementiert werden.
\begin{itemize}
\item DHCP NAP Enforcement
\item IPSec NAP Enforcement 
\item 802.1x NAP Enforcement 
\item VPN NAP Enforcement 
\item TS Gateway NAP Enforcement
\end{itemize}
Die DHCP, IPSec und 802.1x NAP Enforcement Methoden verwalten den direkten Zugriff ins LAN. Die VPN und TS Gateway NAP Enforcement Methoden verwalten den Zugriff von aussen ins LAN. Die Methoden lassen sich miteinander verkn�pfen.
\subsection*{Empfehlung}
Ich empfehle dem Unternehmen die 802.1x NAP Enforcement Methode zu implementieren. Diese Methode ist sehr sicher und einfach zu implementieren. F�r die Implementation dieser Methode werden 802.1x Switches ben�tigt. Die KMU IT Management AG besitzt solche Switches. Die IPSec Variante ist ebenfalls empfehlenswert, jedoch ist die Implementation sehr komplex. Meiner Meinung nach ist die IPSec NAP Enforcement Methode vom Aufwand her nicht geeignet f�r kleinere Firmen.
\end{document}